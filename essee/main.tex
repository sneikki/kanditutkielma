\documentclass{article}

\usepackage[utf8]{inputenc}
\usepackage[finnish]{babel}
\usepackage[babel=true, kerning=fi]{microtype}

\title{Pätkäistyt hashit}
\begin{document}
	\section*{Johdanto}

	\section*{Hajautusarvoista}

	Hajautusarvo on keskeinen työkalu tiedonsiirrossa.

	Esimerkkeinä MD5 ja SHA-1.

	\section*{Pätkäistyt hajautusarvot}

	Kryptografiset hajautusfunktio tuottaa samalla syötteellä saman tuloksen.
	Syö\-tettä prosessoidessa jokainen alkio muuttaa generoitavan hajatusarvon
	tunnistamattomaksi. Tämä ilmiö on tärkeää kryptografisissa sovelluksissa.
	Syötteiden vertailu ei ole mielekästä kryptografisila tiivistefunktioilla
	- identtiset syötteet ovat todettavissa identtisiksi yhtäläisen hajautusarvon perusteella, mutta yhdenkin bitin muutos toisessa syötteessä tuottaa poikkeavan hajautusarvon. Näin ollen syötteet ole todettavissa samoiksi kryptografisella hajatusualgoritmilla, vaikka semanttisessa mielessä syötteet ovat käytännössä samat.

	Tiedostoissa esiintyviä samankaltaisuuksia kutsutaan
	\textit{homologioiksi}. Homologioiden tunnnistaminen vaatii kuitenkin
	toisenlaita lähestymistapaa. Semanttisesti merkityksettömien erojen
	ei ole syytä muuttaa generoitavaa hajautusarvoa tunnistamattomaksi.
	Tätä tavoitetta tukemaan on laadittu kryptografisista hajautusfunktioista
	poikkeava algoritmityyppi. \textit{Pätkäistyt hajautusfunktiot} tulkitsevat
	syötettä tarkastellen alkion kontekstia. Eroavaisuudet syötteissä
	aiheuttavat muutoksia hajautusarvoon ainoastaan paikallisesti,
	ja hajautusarvo säilyy muilta osin lähes tulkoon muuttumattomana.

	Pätkäistyn hajautusarvon generointi tapahtuu yksinkertaistetusti
	kolmessa vaiheessa; syöte jaetaan algoritmikohtaisen logiikan
	perusteella osiin, joille generoidaan hajautusarvo. Lopuksi saaduista
	hajautusarvoista koostetaan palautusarvona annettava pätkäisty
	hajautusarvo. Algoritmi palauttaa siis listan hajautusarvoja, jota
	voidaan vertailla muihin vastaaviin hajautusarvoihin. Homologiat
	ilmenevät pätkäistyssä hajautusarvossa identtisinä hajautusarvoina;
	poikkeavat hajautusarvot viittaavat eroihin syötteiden välillä.

	\section*{CTPH}

	Konteksiriippuvainen tarkastelu voidaan toteuttaa algoritmilla, joka
	syötteen $n$ alkion osiin, joille iteroidaan näistä alkioista riippuva
	hajautusarvo. Tällainen algoritmi toimii kohtalaisesti saman mittaisille
	syötteille, joissa alkioita ainoastaan korvataan toisella kajoamatta
	itse syötteen kokoon. Syötteen koon muuttuessa alkioiden lisättäessä
	tai poistettaessa. Tällöin muiden alkioiden sijainti syötteessä muuttuu,
	ja koko osa siirtyy kokonaan toisen osan vaikutuspiiriin. Nämä
	indeksimuutokset ulottuvat syötteen loppuun asti, jolloin hajautusarvo
	voi syötemuutoksen sijainnista riippuen poiketa merkittävästi
	alkuperäiselle syötteelle generoitavasta hajautusarvosta.

	\textit{Context triggered piecewise hashing (CTPH)} on tekniikka LSH-algoritmien
	toteuttamiseen. Välttääkseen edellä kuvatun ongelman välttääkseen
	CTPH-algoritmit jakavat syötteen vakiokokoisten	osioiden sijaan vaihtelevan kokoisiin osioihin, jotka määräytyvät dynaamisesti syötteen sisällön
	pohjalta. Tällöin alkion lisäys tai poisto vaikuttaa korkeintaan kahden
	osion hajautusarvoon eikä heijastu muihin osioihin.

	CTPH-algoritmeissä osiointi perustuu \textit{laukaisimiin (trigger)}.
	Tämä \textit{rolling hashiksi} kutsuttu hajautusalgoritmi pitää yllä
	hajautusarvoa, joka päivitetään syötettä iteroitaesa jokaisen
	alkion perusteella. Lisäksi hajautusarvoa verrataan algoritmissa
	määriteltyyn laukaisimeen: iteroitaessa aktivoituva laukaisin tarkoittaa
	kontekstin muuttuvan. Vallitsevan osion alkoioille generoidaan
	hajautusarvo, joka lisätään globaaliin hajautusarvoon. Hajautusarvon
	generointiin käytetään tyypillisesti jotakin hajautusfunktiota, kryptografisia (MD5, SHA-1) tai ei-kryptografisia (FNV). Kuvattua menetelmää
	toistaen syötteestä muodostuu yksi hajautusarvo, joka sisältää syötteen
	osiosta generoidut, pääosin toisista riippumattomat alihajautusarvot.

	\subsection*{Toteutuksista}

	CTPH-algoritmien toteutukset poikkeavat toisistaan, ja osa lähestyy
	ongelmaa eri tavoin. Tästä huolimatta erilaiset algoritmit jakavat
	yhteisen perusajatuksen, eli syntaksisten homologioiden löytämisen
	syötteestä.

	\subsection*{spamsum}

	Eräs varhainen CTPH-hajautusfunktio, \textit{spamsum}, tunnistaa
	syötteissä esiintyviä samankaltaisuukia. Spamsum-algoritmin
	totetus perustuu ajatukseen syötteen käsittelystä konteksiriippuvaisesti,
	jolloin syötteestä tarkastellaan tyypillisesti yhden alkion sijaan
	generoidaan riippumaton hajautusarvo jollakin tiivistefunktiolla. Tyypillisesti käytettyjä ovat kryptografiset tiivistefunktiot MD5 sekä SHA. Spamsum käyttää tähän tarkoitukseen yksinkertaista Fowler-Noll-Vo-hajautusfunktiota.

	Spamsumin toiminta pohjautuu edellä esitettyyn yksinkertaistettuun
	kuvaukseen CTPH-algoritmin toiminnasta. Käytännössä Spamsum kuitenkin
	sisältää laajemman totetuksen.


	\subsection*{ssdeep}
	
	\subsection*{SDHASH}
	
	\subsection*{mvHASH-B}

	\section*{Haasteet}

	Pätkäistyjen hajautusfunktioiden todetaan suoriutuvan kiitettävästi.
	Erityisesti SSDEEP-funktion mainitaan kykenevän yhdistämään laajastikin
	muuttettuja sekä puuttellisia tiedostoja alkuperäisiin. Fuzzy hashing ei
	kuitenkaan ole aukoton tekniikka. Nämä algoritmit ovat vaikeasti
	ennustettavia, eikä syötteiden samankaltaisuudelle ole johdettavissa
	yleistä raja-arvoa jonka alittaessaan syötteet eivät ole samankaltaisia.
	Algoritmi laskee numeerisen arvon, jonka tulkinta jää ihmisen vastuulle.
	Tulkintaa vaikeuttaa myös algoritmien syntaksiin perustuvaan analyysiin -
	hajautusarvojen vertailu ei kerro semanttisesta kontekstista lainkaan.

	Datan pakkaus aiheuttaa merkittäviä haasteista pätkäistyille
	hajautusalgoritmeille. Pätkäistyjen hajautusarvojen keskeinen käyttökohde
	on haittaohjelmien tunnistus. Haittaohjelmien levitys sijoittuu
	merkittävissä määrin tietoliikenteeseen, jonka pakatun datan
	analysoinnin automatisointi ei onnistune.

	\section*{Lopetus}

\end{document}

