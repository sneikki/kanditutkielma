\chapter{Johdanto\label{intro}}
 
Tietojenkäsittelyssä käsitellään valtavia määriä dataa. Laskentakyvyn kasvessa myös tiedonsiirto-
ja tallenuskapasiteetti kehittyvät, ja tietojenkäsittelyn kehitys luo myös rikollisia intressejä
edistäviä mahdollisuuksia. Sekä tekninen rikostutkinta että kyberrikollisuus ovat riippuvaisia
tietotekniikan kehityksestä. Uudet innovaatiot synnyttyävät mahdollisuuksia kehittää rikollisia
menetelmiä sekä toisaalta ratkaisuja haitallista toimintaa vastaaan.

Kasvava datavirta helpottaa haittaohjelmien huomaaamatonta leviämistä, ja haittaohjelma saattaa
jäädä kokonaan huomaamatta. Haittaohjelmien torjuntaa voidaan tietoisesti pyrkiä johtamaan harhaan
aiheuttamalla aiheettomia osumia. Haitallisen datan tunnistaminen kasvavasta tietoliikenteestä
edellyttää kohdennettuja resursseja sekä suorituskykyisiä ratkaisuja. Tiedon esittämiselle
kompaktimmassa muodossa onkin tarvetta sen tehokkaan tunnistamisen vuoksi.

Dataa voidaan tiivistää muun muassa pakkaus- ja tiivisteohjelmistoilla. Pakkauksesta poiketen
tyypillisesti käytetyt kryptografiset tiivistefunktiot tiivistävät datan palauttamattomaan muotoon.
Toisaalta tiiviste sisältää lähes uniikin tunnisteen, jolla alkuperäinen data on käytännössä
tunnistettavissa. Tiivistystä voidaankin hyödyntää menetelmänä haittaohjelmien havaitsemisessa
suuresta datamäärästä. Tyypillisesti tunnistetuista haittaohjelmista on luotu tiiviste, jolloin
saman tiivisteeen tuottavaa ohjelmaa voidaan epäillä haittaohjelmaksi. Mikä tahansa muutos tuottaa
kuitenkin tunnistamattomaan tiiviisteen perinteisesti tarkoitukseen käytetyillä kryptografisilla
tiivistefunktioilla. Rikolliset tahot pyrkivät kiertämään haittaohjelmien torjuntamenetelmiä
esimerkiksi tekemällä ohjelmistoon muutoksia. Muutettua haittaohjelmaa ei voida enää tunnistaa
alkuperäiseksi tiivisteitä vertaamalla, vaikka toiminnallisuudeltaan ohjelma olisi identtinen.

Seuraavassa luvussa esitellään yleisesti sumean tiivisteen käsite sekä taustoitetaan eri
tiivistetyyppien toimintaperiaate. Luvussa 3 käydään läpi kolme sumeiden tiivisteiden
prosessointiin käytettyä ohjelmistoa: Ssdeep, Sdhash sekä Mvhash-b. Luvussa 4 käsitellään
sumeiden tiivsteiden soveltamista haittaohjelma-analyysissa edellytyksineen. Lisäksi luvussa
analysoidaan menetelmiä, joilla haittaohjelmien tunnistamista sumein tiivistein voidaan johtaa
harhaan. Luvussa 5 käsitellään sumean tiivistämisen heikkouksia sekä tarkastellaan ohjelmistojen
eroavaisuuksien merkitystä haittaohjelma-analyysissa.
